\section{Further works}
\label{furtherworks}

A thermophysical model of the secondary in the Didymos system has been established. The case study performed covers the extreme thermal inertia range of 50---2000 \si{J.K^{-1}.m^{-2}.s^{-1/2}} with heliocentric distances in the range of 1---1.9 \si{AU}, with and without obliquity. This document also showed the impact of the revolution period on the surface temperatures. Calculations are based on shape models with triangular facets reducted to single points -- medians of the facet -- and their normals. After the previous study of \cite{pelivan}, we implemented the small neglected effects such as the mutual heating from the primary and the self heating. We observed the huge influence that the surface roughness plays on the temperature surfaces.

In addition to the HERA mission, we were asked to participate in the 2016 HO3 mission using our thermophysical model to understand what could be the temperatures to expect from a fast rotation small asteroid.

Ultimately, preliminary AFC viewing simulations have been carried out. An early estimation of the visible area on the secondary of the binary system of asteroids Didymos is now possible.
