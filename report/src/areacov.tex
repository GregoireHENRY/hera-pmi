\section{Surface covering}
\label{sec:8}

In order to get a more visual grasp on HERA's vision on Didymoon, we wanted to make use of the NASA/NAIF Spice database, coupled with Cosmographia.

Spice is an observation geometry information system designed by NASA's Navigation and Ancillary Information Facility (NAIF) to assist scientists in planning and interpreting scientific observations from space-based instruments aboard planetary spacecraft. 

Cosmographia is an interactive tool used to produce 3D visualizations of planet, spacecraft and other objects in the solar system, while taking into account trajectories positions and orientations.

By combining known positions of Didymoon, Didymain, expected positions of HERA and the characteristics of the on-board AFC (Asteroid Framing Camera), we could recreate a visual representation of the visibility HERA will have on Didymoon during the various mission phases. 

We relied on the FOV condition API set of functions (fovray) provided in the NAIF Spice toolkit which verifies if a ray is withing the boundaries of a specified instrument. By using this API, we are theoretically capable of determining which parts of Didymoon are visible at any moment in time. 
